\section{Introduction}
The Internet of Things (IoT) will change humanity in profound and unforeseeable ways\cite{ciscoReport}. According to a report by Cisco, between 2015 and 2020 the number of connected IoT devices is set to double from 25 billion to 50 billion\cite{ciscoReport}. This new revolution is driven by lower prices in manufacturing IoT devices as well as new inventions in radio communication and better processing power. This means more data than ever will be collected and automation will be the norm, rather than the exception. Industrial IoT (IIoT) and smart home are prime example for this. New factories are often operated by only a few workers and lights and temperature are increasingly controlled by an AI rather than the human living in it. This leads to higher productivity, better power management and more comfort among other improvements. In most cases, this new technology is used to aid humans, but recent history has shown that IoT brings many diverse challenges in technology, security and social life.\\
In 2016 a botnet with over 600k infected devices, primarily embedded and IoT devices, overwhelmed several high-profile targets like the DNS provider Dyn with a massive distributed denial-of-service (DDoS) attacks. This caused a temporary outage of their DNS servers making many webpages, among others  Twitter, Spotify und Amazon, temporally inaccessible.\\
The sheer number of IoT devices also makes it important to think about their sustainability. Many IoT devices are cheap to manufacturer and thus only used for a short period of time and are hard to update. This makes basically makes them a throwaway product. Many IoT devices have a stable power connection and are not designed to be energy efficient. \\
IoT devices are by design often close to humans and are intended to analyze what they say, do and how their body reacts. This leads many people to assume that IoT, especially smart home and activity tracking are incompatible with privacy\cite{5Reasons41:online}. In the past many companies overstepped the line what is socially excepted and had to back-roll certain "features"\cite{PrivacyIoT50:online}. For many consumers, privacy is the major concern in smart home\cite{PrivacyIoT50:online}.\\
These four areas, technological improvements, security, sustainability and privacy, are the four areas shaping the development of IoT and the main drivers for many technological as well as architectural decisions in the development process of the product developed as part of this project. The reminder or this report is structured as follows, *** 

\subsection{Motivation}
In this project we develop a new wireless goal for foosball tables. The backend software to manage user data and a primary architecture was part of a previous project and is shortly discussed in the next chapter. The table and the backend software works as intended, but the physical implementation of the table, shown in figure *** is tailor made for the table we used in the development and not portable across different brands and designs. 
\\
SHOW PICTURE OF TABLE!!\\

In order to make our solution attractive to bars and companies alike the solution neeeds to be easy to implement and manage. Thus, the research question for this project is as follow:
\begin{center}
    \textit{What is the optimal design for a scalable wireless foosball goal?}
\end{center}
We will answer the question based on the four aspects, the technology, security, sustainability and privacy. 

\subsection{Delimitations}
Due to time and equipment limitations we will focus on the technology in this report. Also, the findings from our tests are useful as a general reference point when discussing the strength and weaknesses of competing solutions. But as the tests do not follow strict academic rules, these findings are not statistical significant. \\
We also limit the main scope of this report to how we send and receive data from the goal to its controller. We will show the entire architecture but it is not part of the requirements and often not analyze or explained as throughly as the goal sensors.\\
Finally, we did not gather input from experts. This is something we would have wished to do, but was not possible due to time constraints.